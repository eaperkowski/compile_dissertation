\chapter{\textbf{Conclusions}}
\noindent The experiments included in this dissertation test mechanisms that drive patterns expected from photosynthetic least-cost theory across various edaphic and climatic gradients. Specifically, I investigate environmental drivers of carbon costs to acquire nitrogen, tradeoffs between nitrogen and water use, and plant acclimation responses to CO$_2$. These experiments provide important empirical data needed to test assumptions made in optimality models that leverage photosynthetic least-cost frameworks, and are among the first manipulative experiments to show support for patterns expected from theory. Below, I summarize main findings of each chapter, synthesize common patterns observed across experiments, and conclude with a few study ideas that I think will help refine our understanding of plant nutrient acquisition and allocation responses to environmental change leveraging patterns predicted by photosynthetic least-cost theory.

In the first experimental chapter, I quantified carbon costs to acquire nitrogen in a species capable of forming associations with symbiotic nitrogen-fixing bacteria (\textit{Glycine max}) and a species not capable of forming such associations (\textit{Gossypium hirsutum}) grown under four soil nitrogen fertilization treatments and four light availability treatments in a full factorial greenhouse experiment. Supporting hypotheses, increasing light availability increased carbon costs to acquire nitrogen in both species due to a larger increase in belowground carbon biomass than whole plant nitrogen biomass. In further support of hypotheses, increasing fertilization decreased carbon costs to acquire nitrogen due to a larger increase in whole plant nitrogen biomass than belowground carbon biomass. Root nodulation data indicated that \textit{G. max} shifted relative carbon allocation from nitrogen fixation to direct uptake with increasing fertilization, which may explain the reduced responsiveness of \textit{G. max} carbon costs to acquire nitrogen across the fertilization gradient. 

Despite evidence that reductions in the response of \textit{G. max} carbon costs to acquire nitrogen to increasing fertilization may have been driven by shifts away from nitrogen fixation with increasing fertilization, I urge caution in assigning causality to the differential response of carbon costs to acquire nitrogen between species. This is because \textit{G. max} and \textit{G. hirsutum} are not phylogenetically related and have different life histories. Differences in life history between the two species limit my ability to assess whether reductions in the negative effect of increasing fertilization on carbon costs to acquire nitrogen in \textit{G. max} were driven by shifts to direct uptake with increasing fertilization. However, these patterns were later confirmed in the fourth experimental chapter, where similar weaker negative effects of increasing fertilization on carbon costs to acquire nitrogen were observed in \textit{G. max} that were inoculated with symbiotic nitrogen-fixing bacteria compared to \textit{G. max} that were left uninoculated across a similar soil nitrogen fertilization gradient.

In the second experimental chapter, I assessed whether changes in soil nitrogen availability or soil pH drove changes in nitrogen-water use tradeoffs predicted by photosynthetic least-cost theory. I measured leaf traits of mature upper canopy deciduous trees growing in a nine-year nitrogen-by-sulfur field manipulation experiment, where experimental sulfur additions were added with intent to acidify plots. Following patterns expected from the theory, increasing soil nitrogen availability was associated with increased leaf nitrogen content, but not net photosynthesis, resulting in an increase in photosynthetic nitrogen use efficiency. In further support of theory, increasing soil nitrogen availability exhibited slight, but nonsignificant, decreases in leaf $C_\mathrm{i}$:$C_\mathrm{a}$ and increases in measures of photosynthetic capacity. Perhaps the strongest evidence for the theory was a strong negative relationship between leaf nitrogen content and leaf $C_\mathrm{i}$:$C_\mathrm{a}$, of which increased with increasing soil nitrogen availability through a stronger increase in leaf nitrogen content than leaf $C_\mathrm{i}$:$C_\mathrm{a}$.

I found no effect of soil pH on nitrogen-water use tradeoffs aside from a marginal reduction in net photosynthesis rates that marginally reduced photosynthetic nitrogen use efficiency with increasing soil pH. Directionally, reductions in photosynthetic nitrogen use efficiency with increasing soil pH were expected per theory; however, this response was driven by no change in leaf nitrogen content and a reduction in net photosynthesis. Theory predicts that these tradeoffs should be driven by no change in net photosynthesis and an increase in leaf nitrogen content. The general null leaf response to changing soil pH may have been due to experimental treatments directly increased soil nitrogen availability and affected soil pH in opposite patterns, suggesting that soil nitrogen availability may be more important in dictating nitrogen-water use tradeoffs than soil pH per se.

In the third experimental chapter, I quantified variance in leaf nitrogen content across a precipitation and soil resource availability gradient in Texan grasslands. Specifically, I measured area-based leaf nitrogen content, components of area-based leaf nitrogen content (leaf mass per unit leaf area, leaf nitrogen per unit dry biomass), leaf $C_\mathrm{i}$:$C_\mathrm{a}$, and the unit cost of acquiring nitrogen relative to water in 520 individuals comprising 57 species. I found that variance in area-based leaf nitrogen content was positively associated with increasing soil nitrogen availability, soil moisture, vapor pressure deficit, and was negatively related to increasing leaf $C_\mathrm{i}$:$C_\mathrm{a}$. Following patterns expected from theory, a path analysis revealed that the positive soil nitrogen-leaf nitrogen relationship was driven by a positive relationship between soil nitrogen availability and the unit cost of acquiring and using nitrogen relative to water, a positive relationship between the unit cost of acquiring and using nitrogen relative to water, and negative relationship between leaf $C_\mathrm{i}$:$C_\mathrm{a}$ and leaf mass per unit leaf area. Interestingly, there was no effect of $C_\mathrm{i}$:$C_\mathrm{a}$ on leaf nitrogen content per unit dry biomass, indicating that variance in area-based leaf nitrogen content across the environmental gradient was driven by a change in leaf morphology and not leaf chemistry.

In the fourth experimental chapter, I quantified leaf and whole plant acclimation responses in \textit{G. max} grown under two atmospheric CO$_2$ levels, with and without inoculation with \textit{Bradyrhizobium japonicum}, and across nine nitrogen fertilization treatments in a full factorial growth chamber experiment. I found strong evidence that leaf nitrogen content, $V_\mathrm{cmax}$, and $J_\mathrm{max}$ were each downregulated under elevated CO$_2$. A stronger downregulation in $V_\mathrm{cmax}$ than $J_\mathrm{max}$ and stronger downregulation in leaf nitrogen content than $V_\mathrm{cmax}$ or $J_\mathrm{max}$ provided strong support suggesting that leaves were acclimating to elevated CO$_2$ by optimizing leaf photosynthetic resource use efficiency to achieve optimal coordination. In striking support of my hypotheses, I find strong evidence suggesting that leaf acclimation responses to elevated CO$_2$ were decoupled from soil nitrogen fertilization and inoculation treatment, despite apparent strong increases in leaf nitrogen content, $V_\mathrm{cmax}$, and $J_\mathrm{max}$ with increasing fertilization and in inoculated pots. These findings contrast the current formulation of photosynthetic processes in terrestrial biosphere models, where many models simulate downregulations in leaf nitrogen content under elevated CO$_2$ as a function of progressive nitrogen limitation.

There are currently two iterations of optimality models that employ the use of patterns expected from photosynthetic least-cost theory, one for C$_3$ species \shortcite{Wang2017,Smith2019,Stocker2020} and one more recently developed for C$_4$ species \shortcite{Scott2022}. In both model variants, costs to acquire and use nitrogen relative to water are held constant using a global dataset of $\delta^{13}$C \shortcite{Cornwell2018}. Throughout experiments, I show strong evidence suggesting that costs to acquire and use nitrogen are dynamic and vary predictably across environmental gradients, and that changes in these costs scale to alter leaf nitrogen-water use tradeoffs and acclimation responses to changing environments in ways predicted through photosynthetic least-cost theory. Thus, while optimality model simulations show good agreement with measured data \shortcite{Smith2019,Stocker2020}, such models may not be capturing an important source of variability in leaf nitrogen-water use tradeoffs by holding costs of resource use constant across environmental gradients.

First principles of photosynthetic least-cost theory suggest that, in a given environment, plants optimize photosynthesis rates by sacrificing inefficient use of a relatively more abundant (and less costly to acquire) resource for more efficient use of a relatively less abundant (and more costly to acquire) resource. Throughout experimental chapters, I show strong support for these patterns across experiments, where increasing soil nitrogen fertilization generally decreased the cost of acquiring nitrogen relative to water, a pattern that scaled to influence leaf nitrogen-water use tradeoffs. I did not find evidence to suggest that soil moisture influenced nitrogen-water use tradeoffs, though this was due to strong covariation between soil moisture and soil nitrogen availability. Overall, findings across experiments provide empirical validation of photosynthetic least-cost theory needed to further develop optimality models and eventually implement such models in terrestrial biosphere model products. Many terrestrial biosphere model products do not include robust frameworks for simulating acclimation responses to changing environmental conditions, and empirical findings shown here provide some support that optimality models that leverage photosynthetic least-cost theory predictions may improve the ability of terrestrial biosphere models to accurately simulate photosynthetic processes.

Many terrestrial biosphere models predict photosynthetic capacity through plant functional group-specific linear regressions between area-based leaf nitrogen content and $V_\mathrm{cmax}$ \shortcite{Rogers2014,Rogers2017a}, which assumes that leaf nitrogen-photosynthesis relationships are constant across growing environments. I found constant leaf nitrogen-photosynthesis relationships with increasing soil nitrogen availability in the nitrogen-by-sulfur field manipulation experiment. However, results from the CO$_2$-by-nitrogen-by-inoculation manipulation experiment indicated that leaf nitrogen-photosynthesis responses to soil nitrogen availability were dependent on whether nitrogen was limiting. Further investigation regarding the effect of soil nitrogen availability in modifying leaf nitrogen-photosynthesis relationships is warranted to better understand the generality of leaf nitrogen photosynthesis relationships across environmental gradients. However, findings from these experiments suggest that representing photosynthetic processes through positive relationships between soil nitrogen availability, leaf nitrogen, and photosynthetic capacity are likely contributing to erroneous errors in model simulations and may explain the high degree of divergence in simulated processes across terrestrial biosphere models \shortcite{Friedlingstein2014,Davies-Barnard2020}.

The experiments included in this dissertation have provided a strong foundation for me to continue growing as a plant physiological ecologist. I envision five primary avenues for future research that build on the work presented here, which are briefly summarized below:

\begin{enumerate}
    \item Manipulative and environmental gradient experiments included here were designed to provide empirical data needed to test photosynthetic least-cost theory assumptions. While these results show promising patterns for patterns expected from photosynthetic least-cost theory, they do not necessarily address whether these patterns follow those simulated by optimality models that leverage photosynthetic least-cost principles. Thus, a clear future direction of these experiments would be to conduct model-data comparisons using data collected here (or similar experiments) to compare against optimality model simulations.
    
    \item Experiments included here explicitly quantify effects of symbiotic nitrogen fixation on carbon costs to acquire nitrogen, nitrogen-water use tradeoffs, and leaf nitrogen-photosynthesis relationships. However, carbon costs to acquire nitrogen also vary in species that associate with different mycorrhizal types \shortcite{Brzostek2014FUN2,Terrer2018}, and dominant mycorrhizal type in an ecosystem has been shown to determine net biogeochemical cycle dynamics in deciduous forests of the northeastern United States \shortcite{Phillips2013}. Thus, future work should consider conducting similar experiments while manipulating mycorrhizal association to better understand how microbial symbioses modify leaf and whole plant acclimation responses to changing environments. 
    
    \item Recent work indicates a high degree of variance in symbiotic nitrogen fixation rates across terrestrial biosphere models \shortcite{Meyerholt2016,Davies-Barnard2020}, perhaps due to nitrogen fixation rates that are implemented across terrestrial biosphere models as a function of temperature \shortcite{Houlton2008}. While energetic costs of nitrogen fixation are dependent on temperature, I show that structural carbon costs to acquire nitrogen via symbiotic nitrogen fixation are driven by factors that influence demand to acquire nitrogen (i.e. CO$_2$, light) and are modified by soil nitrogen supply. The light-by-nitrogen greenhouse experiment was published in \textit{Journal of Experimental Botany}, and a reviewer encouraged future work to include a model-data comparison comparing structural carbon costs to acquire nitrogen measured in the experiment to carbon costs to acquire nitrogen simulated by the FUN biogeochemical model \shortcite{Fisher2010,Brzostek2014FUN2,Allen2020}. Conveniently, FUN calculates carbon costs to acquire nitrogen following the same calculation used in the first and fourth experimental chapter. Conducting such a model-data comparison would be a useful step toward identifying biases in the FUN biogeochemical model, which is currently coupled to several terrestrial biosphere models \shortcite{Clark2011,Shi2016,Lawrence2019,Davies-Barnard2020}.

    \item Carbon costs to acquire nitrogen relative to water were quantified at the leaf level as a function of $\delta^{13}$C and vapor pressure deficit, while structural carbon costs to acquire nitrogen were quantified at the whole plant level as the ratio of belowground carbon allocation per unit whole plant nitrogen biomass. As increasing soil nitrogen availability decreases both leaf and whole plant estimates of costs to acquire and use nitrogen, one might expect leaf and whole plant carbon cost to acquire nitrogen estimates to covary. Future work should consider investigating if leaf and whole plant estimates of carbon costs to acquire nitrogen covary and evaluate whether environmental conditions (or species acquisition strategy) modifies any of this possible covariance. Strong covariance between leaf and whole plant costs of nitrogen acquisition could be a possible avenue to implement frameworks for allowing costs of nitrogen acquisition to vary in optimality models, as the FUN model calculates carbon costs of nitrogen acquisition at the whole plant level.
    
    \item While experiments included here target effects of soil nitrogen availability on carbon costs to acquire nitrogen and associated leaf nitrogen-water use tradeoffs, photosynthetic least-cost theory predicts that plants acclimate their photosynthetic processes by minimizing the summed cost of nutrient (not just nitrogen) and water use. Therefore, the theory would predict similar leaf acclimation responses across soil phosphorus or other nutrient availability gradients. Recent iterations of the FUN biogeochemical cycle includes a framework for determining the carbon and nitrogen cost of acquiring and using phosphorus, which similarly varies in species with different nutrient acquisition strategies \shortcite{Allen2020}. The implementation of this model in a terrestrial biosphere model (E3SM) was also recently shown to improve model performance of ecosystem nutrient limitation \shortcite{Braghiere2022}. As nitrogen and phosphorus commonly co-limit leaf photosynthesis and primary productivity, extending experiments reported here to investigate carbon and nitrogen costs of phosphorus use, and whether these patterns scale to leaf nutrient-water use tradeoffs would be a useful next step in understanding extensions and limitations of photosynthetic least-cost theory.
\end{enumerate}

\noindent The experiments included in this dissertation and the proposed experiments summarized above provide a snapshot view of the things that I have learned throughout my time as a graduate student. I am excited to continue learning and growing as a plant ecophysiologist, ecologist, and scientist, and look forward to continuing along my journey of investigating nutrient acquisition and allocation responses to global change.