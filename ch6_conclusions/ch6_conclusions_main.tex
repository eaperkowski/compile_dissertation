\chapter{\textbf{Conclusions}}
\noindent The experiments included in this dissertation were designed to test mechanisms that drive patterns expected from photosynthetic least-cost theory across various edaphic and climatic gradients. Specifically, I evaluate the context dependency of carbon costs to acquire nitrogen across soil nitrogen availability and how variance in carbon costs to acquire nitrogen scales to influence leaf and whole plant acclimation responses to changing environments. Below, I summarize each experimental chapter, briefly synthesize main patterns observed across experiments, and conclude this dissertation with a few proposed experiments that will further our understanding of mechanisms governing patterns expected from photosynthetic least-cost theory.

In the first experimental chapter, I quantified carbon costs to acquire nitrogen in a species capable of forming associations with symbiotic nitrogen-fixing bacteria (\textit{Glycine max}) and a species not capable of forming such associations (\textit{Gossypium hirsutum}) grown under four soil nitrogen fertilization treatments and four light availability treatments in a full factorial greenhouse experiment. I found that increasing light availability increased carbon costs to acquire nitrogen in both species due to a larger increase in belowground carbon biomass than whole plant nitrogen biomass. These patterns were observed in both species. I also found that increasing fertilization decreased carbon costs to acquire nitrogen due to a larger increase in whole plant nitrogen biomass than belowground carbon biomass. While these patterns were observed in both species, carbon costs to acquire nitrogen in \textit{G. max} were less responsive to increasing fertilization than \textit{G. hirsutum}, providing some support for my second hypothesis. Root nodulation data indicated that \textit{G. max} shifted relative carbon allocation from nitrogen fixation to direct uptake with increasing fertilization, which may explain the reduced responsiveness of \textit{G. max} carbon costs to acquire nitrogen across the fertilization gradient. 

Despite evidence that reductions in the response of \textit{G. max} carbon costs to acquire nitrogen to increasing fertilization may have been driven by shifts away from nitrogen fixation with increasing fertilization, I urge caution in assigning causality to these results. This is because \textit{G. max} and \textit{G. hirsutum} are not phylogenetically related and have different life histories. Specifically,\textit{G. max} is a herbaceous annual species, while \textit{G. hirsutum} is a woody perennial species. Differences in life history between the two species limit my ability to assess whether reductions in the negative effect of increasing fertilization on carbon costs to acquire nitrogen in \textit{G. max} were driven by shifts to direct uptake with increasing fertilization. However, these patterns were later confirmed in the fourth experimental chapter, where I quantify similar weaker negative effects of increasing fertilization on carbon costs to acquire nitrogen in \textit{G. max} that were inoculated with symbiotic nitrogen-fixing bacteria compared to \textit{G. max} that were left uninoculated across a similar soil nitrogen fertilization gradient.

In the second experimental chapter, I quantified a series of leaf traits to assess whether changes in soil nitrogen availability or soil pH drove changes in nitrogen-water use tradeoffs predicted by photosynthetic least-cost theory. I measured leaf traits of mature upper canopy deciduous trees growing in a nine-year nitrogen-by-sulfur field manipulation experiment, where experimental sulfur additions were added with intent to acidify plots. Following patterns expected from the theory, increasing soil nitrogen availability was associated with increased leaf nitrogen content, but not net photosynthesis, resulting in an increase in photosynthetic nitrogen use efficiency. In further support of theory, increasing soil nitrogen availability exhibited slight, but nonsignificant, decreases in leaf $C_\mathrm{i}$:$C_\mathrm{a}$ and increases in measures of photosynthetic capacity. Perhaps the strongest evidence for the theory was a strong negative relationship between leaf nitrogen content and leaf $C_\mathrm{i}$:$C_\mathrm{a}$, of which increased with increasing soil nitrogen availability through a stronger increase in leaf nitrogen content than leaf $C_\mathrm{i}$:$C_\mathrm{a}$.

Interestingly, I found no effect of soil pH on nitrogen-water use tradeoffs aside from a marginal reduction in net photosynthesis rates that marginally reduced photosynthetic nitrogen use efficiency with increasing soil pH. Directionally, reductions in photosynthetic nitrogen use efficiency with increasing soil pH are expected per theory; however, this was driven by no change in leaf nitrogen content and a reduction in net photosynthesis. Theory predicts that these tradeoffs should be driven by no change in net photosynthesis and an increase in leaf nitrogen content. Regardless, the general null leaf response to changing soil pH may have been due to experimental treatments directly increased soil nitrogen availability and affected soil pH in opposite patterns, suggesting that soil nitrogen availability may be more important in dictating nitrogen-water use tradeoffs than soil pH per se.

In the third experimental chapter, I quantified variance in leaf nitrogen content across a precipitation and soil resource availability gradient in Texan grasslands. Specifically, I measured area-based leaf nitrogen content, components of area-based leaf nitrogen content (leaf mass per unit leaf area, leaf nitrogen per unit dry biomass), leaf $C_\mathrm{i}$:$C_\mathrm{a}$, and the unit cost of acquiring nitrogen relative to water in 520 individuals comprising 57 species. I found that variance in area-based leaf nitrogen content was negatively associated with increasing soil nitrogen availability, soil moisture, vapor pressure deficit, and was negatively related to increasing leaf $C_\mathrm{i}$:$C_\mathrm{a}$. Following patterns expected from theory, a path analysis revealed that the positive soil nitrogen-leaf nitrogen relationship was driven by a positive relationship between soil nitrogen availability and the unit cost of acquiring and using nitrogen relative to water, a positive relationship between the unit cost of acquiring and using nitrogen relative to water, and negative relationship between leaf $C_\mathrm{i}$:$C_\mathrm{a}$ and leaf mass per unit leaf area. Interestingly, there was no effect of $C_\mathrm{i}$:$C_\mathrm{a}$ on leaf nitrogen content per unit dry biomass, indicating that variance in area-based leaf nitrogen content across the environmental gradient was driven by a change in leaf morphology and not leaf chemistry. 

In the fourth experimental chapter, I quantified leaf and whole plant acclimation responses in \textit{G. max} grown under two atmospheric CO$_2$ levels, with and without inoculation with \textit{Bradyrhizobium japonicum}, and across nine nitrogen fertilization treatments in a full factorial growth chamber experiment. I found strong evidence that leaf nitrogen content, $V_\mathrm{cmax}$, and $J_\mathrm{max}$ were each downregulated under elevated CO$_2$. A stronger downregulation in $V_\mathrm{cmax}$ than $J_\mathrm{max}$ and stronger downregulation in leaf nitrogen content than $V_\mathrm{cmax}$ or $J_\mathrm{max}$ provided strong support suggesting that leaves were acclimating to elevated CO$_2$ by optimizing leaf photosynthetic resource use efficiency to achieve optimal coordination. In striking support of my hypotheses, I find strong evidence suggesting that leaf acclimation responses to elevated CO$_2$ were decoupled from soil nitrogen fertilization and inoculation treatment, despite apparent strong increases in leaf nitrogen content, $V_\mathrm{cmax}$, and $J_\mathrm{max}$ with increasing fertilization and in inoculated pots. These findings contrast the current formulation of photosynthetic processes in terrestrial biosphere models, where many models simulate downregulations in leaf nitrogen content under elevated CO$_2$ schemes as a function of progressive nitrogen limitation.

There are currently two iterations of optimality models that employ the use of patterns expected from photosynthetic least-cost theory, one for C$_3$ species \shortcite{Wang2017,Smith2019,Stocker2020} and one more recently developed for C$_4$ species \shortcite{Scott2022}. In both model variants, costs to acquire and use nitrogen relative to water are held constant using a global dataset of $\delta^{13}$C \shortcite{Cornwell2018}. The C$_3$ optimality model initially assumed a constant cost to acquire and use nitrogen relative to water value of 240 \shortcite{Wang2017}, later corrected to 146 \shortcite{Stocker2020}, while the C$_4$ optimality model assumes a constant value of 166 \shortcite{Scott2022}.

The light-by-nitrogen greenhouse experiment and CO$_2$-by-nitrogen-by- inoculation manipulation experiment revealed that carbon costs to acquire nitrogen increased due to factors that influence plant nutrient demand (i.e. light availability and atmospheric CO$_2$), and decreased with increasing soil nitrogen availability. Additionally, the unit cost of acquiring nitrogen relative to water was negatively related to increasing soil nitrogen availability across Texan grasslands. Patterns from these experiments demonstrate strong evidence that the unit cost of acquiring and using nitrogen relative to water is a dynamic trait that changes across environmental gradients. The current inclusion of a constant parameterized value in current iterations of optimality models therefore likely contributes to erroneous errors in model simulations, indicating a need for future optimality model development to prioritize frameworks for representing unit costs of acquiring and using nitrogen relative to water. The experiments conducted here provide important empirical data needed to evaluate these frameworks.

I observed strong leaf nitrogen-water use tradeoffs in the nitrogen-by-pH field manipulation, Texas grassland environmental gradient, and CO$_2$-by-nitrogen-by-inoculation manipulation experiments. These experiments each provide strong support for first principles of photosynthetic least-cost theory first proposed in \shortciteN{Wright2003}. Across experiments, leaf nitrogen-water use tradeoffs were modified by soil nitrogen availability, atmospheric CO$_2$, and vapor pressure deficit.


Species identity strongly determines the net effect of soil nitrogen availability and climate on leaf physiology

Future work

Ode to learning
I conclude this dissertation by reflecting on my time as a Ph.D. student at Texas Tech University, specifically in the Department of Biological Sciences and under the mentorship of Dr. Nick Smith. I began this research in Fall 2018 with limited knowledge in plant ecophysiology.