\chapter{\textbf{Conclusions}}

Experiments included in this dissertation leverage patterns expected from photosynthetic least-cost theory to investigate effects of soil resource availability and aboveground climate on costs of nitrogen acquisition, leaf nitrogen-water use tradeoffs, and plant acclimation responses to elevated CO$_2$. Photosynthetic least-cost theory provides a contemporary framework for understanding impacts of climatic and edaphic characteristics on plant ecophysiological processes, namely leaf nitrogen allocation and photosynthetic capacity. When I began planning experiments for this dissertation in August 2018,, empirical tests of the theory were sparse and model development was just beginning with a goal of eventually implementing the theory in terrestrial biosphere models. At the time, it was critical that experimentation be done to test underlying assumptions of the theory and validate its suitability for implementing in terrestrial biosphere models.

Early iterations of model development held the unit cost of acquiring nitrogen relative to water constant \shortcite{Wang2017}, in part because limited data existed to evaluate how this parameter changes across spatiotemporal scales and different environmental gradients. However, the Fixation and Uptake of Nitrogen model \shortcite{Fisher2010,Brzostek2014FUN2} indicates that costs of nitrogen acquisition decreased with increasing soil nitrogen availability and varies in species with different nitrogen acquisition strategies, suggesting that the unit cost of acquiring nitrogen relative to water should change across nitrogen availability gradients. Additionally, \



All experimental chapters in this dissertation provide strong and consistent support for patterns expected from the theory across different experimental approaches, spatiotemporal scales, and different plant functional groups. In this chapter, I first summarize experimental approaches and primary findings of each experimental chapter. Then, I use findings from the four experimental chapters to synthesize recommendations for future photosynthetic least-cost theory model development, and propose experiments that will allow for further understanding of mechanisms that drive patterns expected from photosynthetic least-cost theory across environmental gradients.