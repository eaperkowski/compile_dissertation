\chapter{\textbf{Conclusions}}
\noindent The experiments described in this dissertation were designed to test mechanisms that drive patterns expected from photosynthetic least-cost theory across various soil resource availability and climatic gradients. The first experimental experiment evaluated variance in carbon costs to acquire nitrogen in \textit{Glycine max} and \textit{Gossypium hirsutum} grown across four nitrogen fertilization treatments and four light availability treatments in a full factorial greenhouse experiment. The second experiment investigated nitrogen-water use tradeoffs across a soil nitrogen availability and soil pH gradient in mature deciduous tree species growing in a nine-year nitrogen-by-sulfur field manipulation experiment. The third experiment explored variance in costs to acquire nitrogen relative to water, leaf $C_\mathrm{i}$:$C_\mathrm{a}$, and components of leaf nitrogen content in species scattered along a precipitation and soil nitrogen availability gradient in open canopy grasslands of Texas. Finally, the fourth experiment quantified leaf and whole plant acclimation responses to elevated CO$_2$ in \textit{G. max} grown under nine soil nitrogen fertilization treatments and two inoculation treatments in a full factorial growth chamber experiment. Below, I provide a brief summary of major findings from each experiment, synthesize common patterns observed across experiments, interpret major findings in the context of photosynthetic least-cost theory and propose directions for future model development, and conclude this dissertation with suggestions for future manipulative and environmental gradient experiments that will allow us to better understand mechanisms that drive patterns expected from photosynthetic least-cost theory.

\section{\textit{Experiment summaries}}
\subsection{\textit{Light-by-nitrogen greenhouse experiment}}
\noindent The first experimental chapter in this dissertation sought to understand how carbon costs to acquire nitrogen vary across soil nitrogen and light availability gradients in two species with different nutrient acquisition strategies. \textit{Glycine max} is a legume capable of acquiring nitrogen via direct uptake pathways, through associations with symbiotic nitrogen-fixing bacteria, and is capable of forming associations with arbuscular mycorrhizal fungi. \textit{Gossypium hirsutum} is also capable of acquiring nitrogen via direct uptake pathways and through associations with arbuscular mycorrhizal fungi, but is not able to form associations with symbiotic nitrogen-fixing bacteria. Regardless of species, I found strong evidence linking increasing fertilization with reductions in carbon costs to acquire nitrogen, a pattern that was driven by a larger increase in nitrogen uptake than belowground carbon investment, suggesting that increasing fertilization allowed plants to increase nitrogen-uptake efficiency \shortcite{Lu2022}. I also find strong evidence linking an increase in light availability to an increase in carbon costs to acquire nitrogen, which was presumably a response driven by an increase in demand to allocate nitrogen to photosynthetic enzymes with increasing fertilization.

Interestingly, the first chapter also indicates that carbon costs to acquire nitrogen in \textit{G. max} were generally less responsive to changes in nitrogen fertilization than \textit{G. hirsutum}. These findings were observed in coordination with a strong reduction in root nodulation with increasing fertilization, indicating that \textit{G. max} were likely shifting away from investment in symbiotic nitrogen fixation and toward direct uptake as costs to acquire nitrogen via direct uptake became more similar with increasing fertilization. These patterns follow resource optimization theory \shortcite{Rastetter2001}, where individuals are expected to maximize nutrient uptake efficiency in a given environment, which could be achieved by preferentially investing in the nutrient acquisition pathway that maximizes nutrient returns from a given carbon investment. Assigning causality to these patterns is challenging, as \textit{G. max} and \textit{G. hirsutum} differ in their growth forms and growth durations (\textit{G. max} is a herbaceous annual while \textit{G. hirsutum} is a woody perennial) and are not phylogenetically related. Differences in life history between the two species limited my ability to assess whether reductions in the negative effect of increasing fertilization on carbon costs to acquire nitrogen in \textit{G. max} were driven by shifts to direct uptake with increasing fertilization. However, these patterns were later validated in the fourth experimental chapter, where I quantify similar weaker negative effects of increasing fertilization on carbon costs to acquire nitrogen in \textit{G. max} that were inoculated with symbiotic nitrogen-fixing bacteria compared to \textit{G. max} that were left uninoculated across a similar soil nitrogen fertilization gradient.

\subsection{\textit{Nitrogen-by-pH field manipulation experiment}}
