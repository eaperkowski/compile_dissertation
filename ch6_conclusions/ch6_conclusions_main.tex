\chapter{\textbf{Conclusions}}
\noindent The experiments included in this dissertation were designed to test mechanisms that drive patterns expected from photosynthetic least-cost theory across various edaphic and climatic gradients. Specifically, I evaluate the context dependency of carbon costs to acquire nitrogen across soil nitrogen availability and how variance in carbon costs to acquire nitrogen scales to influence leaf and whole plant acclimation responses to changing environments.

In the first experimental chapter, I quantified carbon costs to acquire nitrogen in a species capable of forming associations with symbiotic nitrogen-fixing bacteria (\textit{Glycine max}) and a species not capable of forming such associations (\textit{Gossypium hirsutum}) grown under four soil nitrogen fertilization treatments and four light availability treatments in a full factorial greenhouse experiment. I found that increasing light availability increased carbon costs to acquire nitrogen in both species due to a larger increase in belowground carbon biomass than whole plant nitrogen biomass. These patterns were observed in both species. I also found that increasing fertilization decreased carbon costs to acquire nitrogen due to a larger increase in whole plant nitrogen biomass than belowground carbon biomass. While these patterns were observed in both species, carbon costs to acquire nitrogen in \textit{G. max} were less responsive to increasing fertilization than \textit{G. hirsutum}, providing some support for my second hypothesis. Root nodulation data indicated that \textit{G. max} shifted relative carbon allocation from nitrogen fixation to direct uptake with increasing fertilization, which may explain the reduced responsiveness of \textit{G. max} carbon costs to acquire nitrogen across the fertilization gradient. 

Despite evidence that reductions in the response of \textit{G. max} carbon costs to acquire nitrogen to increasing fertilization may have been driven by shifts away from nitrogen fixation with increasing fertilization, I urge caution in assigning causality to the differential response of carbon costs to acquire nitrogen between species. This is because \textit{G. max} and \textit{G. hirsutum} are not phylogenetically related and have different life histories. Specifically,\textit{G. max} is a herbaceous annual species, while \textit{G. hirsutum} is a woody perennial species. Differences in life history between the two species limit my ability to assess whether reductions in the negative effect of increasing fertilization on carbon costs to acquire nitrogen in \textit{G. max} were driven by shifts to direct uptake with increasing fertilization. However, these patterns were later confirmed in the fourth experimental chapter, where I quantify similar weaker negative effects of increasing fertilization on carbon costs to acquire nitrogen in \textit{G. max} that were inoculated with symbiotic nitrogen-fixing bacteria compared to \textit{G. max} that were left uninoculated across a similar soil nitrogen fertilization gradient.

In the second experimental chapter, I assessed whether changes in soil nitrogen availability or soil pH drove changes in nitrogen-water use tradeoffs predicted by photosynthetic least-cost theory. I measured leaf traits of mature upper canopy deciduous trees growing in a nine-year nitrogen-by-sulfur field manipulation experiment, where experimental sulfur additions were added with intent to acidify plots. Following patterns expected from the theory, increasing soil nitrogen availability was associated with increased leaf nitrogen content, but not net photosynthesis, resulting in an increase in photosynthetic nitrogen use efficiency. In further support of theory, increasing soil nitrogen availability exhibited slight, but nonsignificant, decreases in leaf $C_\mathrm{i}$:$C_\mathrm{a}$ and increases in measures of photosynthetic capacity. Perhaps the strongest evidence for the theory was a strong negative relationship between leaf nitrogen content and leaf $C_\mathrm{i}$:$C_\mathrm{a}$, of which increased with increasing soil nitrogen availability through a stronger increase in leaf nitrogen content than leaf $C_\mathrm{i}$:$C_\mathrm{a}$.

I found no effect of soil pH on nitrogen-water use tradeoffs aside from a marginal reduction in net photosynthesis rates that marginally reduced photosynthetic nitrogen use efficiency with increasing soil pH. Directionally, reductions in photosynthetic nitrogen use efficiency with increasing soil pH were as expected per theory; however, this response was driven by no change in leaf nitrogen content and a reduction in net photosynthesis. Theory predicts that these tradeoffs should be driven by no change in net photosynthesis and an increase in leaf nitrogen content. Regardless, the general null leaf response to changing soil pH may have been due to experimental treatments directly increased soil nitrogen availability and affected soil pH in opposite patterns, suggesting that soil nitrogen availability may be more important in dictating nitrogen-water use tradeoffs than soil pH per se.

In the third experimental chapter, I quantified variance in leaf nitrogen content across a precipitation and soil resource availability gradient in Texan grasslands. Specifically, I measured area-based leaf nitrogen content, components of area-based leaf nitrogen content (leaf mass per unit leaf area, leaf nitrogen per unit dry biomass), leaf $C_\mathrm{i}$:$C_\mathrm{a}$, and the unit cost of acquiring nitrogen relative to water in 520 individuals comprising 57 species. I found that variance in area-based leaf nitrogen content was positively associated with increasing soil nitrogen availability, soil moisture, vapor pressure deficit, and was negatively related to increasing leaf $C_\mathrm{i}$:$C_\mathrm{a}$. Following patterns expected from theory, a path analysis revealed that the positive soil nitrogen-leaf nitrogen relationship was driven by a positive relationship between soil nitrogen availability and the unit cost of acquiring and using nitrogen relative to water, a positive relationship between the unit cost of acquiring and using nitrogen relative to water, and negative relationship between leaf $C_\mathrm{i}$:$C_\mathrm{a}$ and leaf mass per unit leaf area. Interestingly, there was no effect of $C_\mathrm{i}$:$C_\mathrm{a}$ on leaf nitrogen content per unit dry biomass, indicating that variance in area-based leaf nitrogen content across the environmental gradient was driven by a change in leaf morphology and not leaf chemistry.

In the fourth experimental chapter, I quantified leaf and whole plant acclimation responses in \textit{G. max} grown under two atmospheric CO$_2$ levels, with and without inoculation with \textit{Bradyrhizobium japonicum}, and across nine nitrogen fertilization treatments in a full factorial growth chamber experiment. I found strong evidence that leaf nitrogen content, $V_\mathrm{cmax}$, and $J_\mathrm{max}$ were each downregulated under elevated CO$_2$. A stronger downregulation in $V_\mathrm{cmax}$ than $J_\mathrm{max}$ and stronger downregulation in leaf nitrogen content than $V_\mathrm{cmax}$ or $J_\mathrm{max}$ provided strong support suggesting that leaves were acclimating to elevated CO$_2$ by optimizing leaf photosynthetic resource use efficiency to achieve optimal coordination. In striking support of my hypotheses, I find strong evidence suggesting that leaf acclimation responses to elevated CO$_2$ were decoupled from soil nitrogen fertilization and inoculation treatment, despite apparent strong increases in leaf nitrogen content, $V_\mathrm{cmax}$, and $J_\mathrm{max}$ with increasing fertilization and in inoculated pots. These findings contrast the current formulation of photosynthetic processes in terrestrial biosphere models, where many models simulate downregulations in leaf nitrogen content under elevated CO$_2$ schemes as a function of progressive nitrogen limitation.

There are currently two iterations of optimality models that employ the use of patterns expected from photosynthetic least-cost theory, one for C$_3$ species \shortcite{Wang2017,Smith2019,Stocker2020} and one more recently developed for C$_4$ species \shortcite{Scott2022}. In both model variants, costs to acquire and use nitrogen relative to water are held constant using a global dataset of $\delta^{13}$C \shortcite{Cornwell2018}. The C$_3$ optimality model initially assumed a constant cost to acquire and use nitrogen relative to water value of 240 \shortcite{Wang2017}, later corrected to 146 \shortcite{Stocker2020}, while the C$_4$ optimality model assumes a constant value of 166 \shortcite{Scott2022}. Throughout experiments, I show strong evidence suggesting that costs to acquire and use nitrogen are dynamic and vary predictably across environmental gradients, and that changes in these costs yield predictable changes in leaf nitrogen-water use tradeoffs and acclimation responses to changing environments. Thus, optimality models that hold unit costs of resource use constant may contribute to erroneous errors in model simulations. Future iterations of optimality models that leverage patterns expected from photosynthetic least-cost theory should consider development of explicit schemes for dynamically calculating costs to acquire and use nitrogen relative to water, or be coupled with previously established plant nitrogen uptake models (e.g., FUN) \shortcite{Fisher2010,Brzostek2014FUN2,Allen2020}.

First principles of photosynthetic least-cost theory suggest that plants can optimize photosynthesis rates by sacrificing inefficient use of a relatively more abundant (and less costly to acquire) resource for more efficient use of a relatively less abundant (and more costly to acquire) resource. I show strong support for these patterns across experiments, where increasing soil nitrogen fertilization generally decreased the cost of acquiring nitrogen relative to water, a pattern that scaled to influence leaf nitrogen-water use tradeoffs. These findings provide important empirical validation of photosynthetic least-cost theory needed to further develop optimality models and eventually implement in terrestrial biosphere model products. Many current terrestrial biosphere model products do not include robust frameworks for simulating acclimation responses to changing environmental conditions, and empirical findings shown here provide some support that optimality models that leverage photosynthetic least-cost theory predictions may improve the ability of terrestrial biosphere models to accurately simulate photosynthetic processes. Future work should leverage data collected from these experiments, particularly the environmental gradient experiment across Texan grasslands, to conduct model-data comparisons to evaluate optimality model performance.

Many terrestrial biosphere models predict photosynthetic capacity through plant functional group-specific linear regressions between area-based leaf nitrogen content and $V_\mathrm{cmax}$ \shortcite{Rogers2014,Rogers2017a}, which assumes that leaf nitrogen-photosynthesis relationships are constant across growing environments. I found constant leaf nitrogen-photosynthesis relationships with increasing soil nitrogen availability in the nitrogen-by-sulfur field manipulation experiment. However, results from the CO$_2$-by-nitrogen-by-inoculation manipulation experiment indicated that leaf nitrogen-photosynthesis responses to soil nitrogen availability were dependent on whether nitrogen was limiting. Specifically, similar increases in area-based leaf nitrogen content, $V_\mathrm{cmax25}$, and $J_\mathrm{max25}$ with increasing fertilization resulted in no change in the fraction of leaf nitrogen allocated to photosynthesis in uninoculated pots, while larger increases in area-based leaf nitrogen content than $V_\mathrm{cmax25}$ and $J_\mathrm{max25}$ with increasing fertilization decreased the fraction of leaf nitrogen allocated to photosynthesis in inoculated pots. As inoculated pots were able to access less finite supply of nitrogen across the fertilization gradient, these patterns suggest that constant leaf nitrogen-photosynthesis relationships may only apply in environments where nitrogen is limiting. Further investigation is certainly warranted regarding the effect of soil nitrogen availability in modifying leaf nitrogen-photosynthesis relationships, but findings from these experiments suggest that representing photosynthetic processes through positive relationships between soil nitrogen availability, leaf nitrogen, and photosynthetic capacity are likely contributing to erroneous errors in model simulations and may be an explanation for the high degree of divergence between carbon and nutrient flux simulations across terrestrial biosphere model products \shortcite{Friedlingstein2014,Davies-Barnard2020}.

The experiments included in this dissertation have provided a strong foundation for me to continue growing as a plant physiological ecologist. I envision five primary avenues for future research that build on the work presented here, which are briefly summarized below:

\begin{enumerate}
    \item Manipulative and environmental gradient experiments included in this dissertation were designed to provide empirical data needed to test photosynthetic least-cost theory assumptions. While these results show promising patterns for patterns expected from photosynthetic least-cost theory, they do not necessarily address whether these patterns follow those simulated by optimality models that leverage photosynthetic least-cost principles. Thus, a clear future direction of this research could be to conduct model-data comparisons using data collected here (or similar experiments) to compare against optimality model simulations.
    
    \item Experiments included in this dissertation explicitly quantify effects of symbiotic nitrogen fixation on carbon costs to acquire nitrogen, nitrogen-water use tradeoffs, and leaf nitrogen-photosynthesis relationships. However, carbon costs to acquire nitrogen also vary in species that associate with different mycorrhizal types \shortcite{Brzostek2014FUN2,Terrer2018}, and dominant mycorrhizal type in an ecosystem may dictate net biogeochemical cycle dynamics \shortcite{Phillips2013}. Thus, future work should consider conducting similar experiments while manipulating mycorrhizal association to comprehensively understand how microbial symbioses modify leaf and whole plant acclimation responses to changing environments. This avenue of research would be particularly useful in forested ecosystems, as previous work suggests that dominant mycorrhizal type in hardwood forests dictate net biogeochemical cycle dynamics
    
    \item Recent work indicates a high degree of variance in symbiotic nitrogen fixation rates across terrestrial biosphere models \shortcite{Davies-Barnard2020,Meyerholt2016}, perhaps due to nitrogen fixation rates that are implemented across terrestrial biosphere models as a function of temperature \shortcite{Houlton2008}. While energetic costs of nitrogen fixation are certainly temperature dependent, I show that structural costs of nitrogen fixation are driven by shifts in soil resource availability. The light-by-nitrogen greenhouse experiment was recently published in \textit{Journal of Experimental Botany}, and a reviewer encouraged future work to include a model-data comparison comparing carbon costs to acquire nitrogen measured in the experiment to carbon costs to acquire nitrogen simulated by the FUN biogeochemical model \shortcite{Fisher2010,Brzostek2014FUN2,Allen2020}. Conveniently, FUN calculates carbon costs to acquire nitrogen following the same calculation used in the first and fourth experimental chapter, and doing this would be a useful next step toward understanding why nitrogen fixation simulations in terrestrial biosphere models might deviate to such a large degree between products.

    \item Carbon costs to acquire nitrogen relative to water were quantified at the leaf level as a function of $\delta^{13}$C and vapor pressure deficit, while structural carbon costs to acquire nitrogen were quantified at the whole plant level as the ratio of belowground carbon allocation per unit whole plant nitrogen biomass. As increasing soil nitrogen availability decreases both leaf and whole plant estimates of costs to acquire and use nitrogen, one might expect leaf and whole plant carbon cost to acquire nitrogen estimates to covary. Future work should consider investigating if leaf and whole plant estimates of carbon costs to acquire nitrogen covary and evaluate whether environmental conditions (or species acquisition strategy) modifies any of this possible covariance. Strong covariance between leaf and whole plant costs of nitrogen acquisition could be a possible avenue to implement frameworks for allowing costs of nitrogen acquisition to vary in optimality models, as the FUN model calculates carbon costs of nitrogen acquisition at the whole plant level.
    
    \item While experiments included in this dissertation target effects of soil nitrogen availability on carbon costs to acquire nitrogen and associated leaf nitrogen-water use tradeoffs, photosynthetic least-cost theory predicts that costs of nutrient use, not just nitrogen, relative to water are substitutable. Recent iterations of the FUN biogeochemical cycle includes the carbon and nitrogen cost of acquiring and using phosphorus, which similarly varies in species with different nutrient acquisition strategies \shortcite{Allen2020}. The implementation of this model in a terrestrial biosphere model (E3SM) was recently shown to improve model performance of ecosystem nutrient limitation \shortcite{Braghiere2022}. As phosphorus commonly co-limits leaf photosynthesis and primary productivity, extending experiments reported here to investigate carbon and nitrogen costs of phosphorus use may be a useful next step in understanding extensions and limitations of photosynthetic least-cost theory.
\end{enumerate}

I conclude this dissertation with a brief word of thanks to all who have shaped me into the plant physiological ecologist that I am today. Specifically, I am thankful for the incredible mentorship of my advisor and committee chair, Dr. Nick Smith, who provided invaluable insight for each of these experimental chapters, and for my committee members for their helpful advise and support throughout these experiments. I am excited to continue growing as a plant physiological ecologist, look forward to continuing to understand nutrient acquisition and allocation responses to global change, and am excited to help mentor future generations of young researchers.