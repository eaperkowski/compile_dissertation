\begin{singlespace}
    \chapter{\textbf{Soil nitrogen availability modifies leaf nitrogen economies in mature temperate deciduous forests: a direct test of photosynthetic least-cost theory}}
    \end{singlespace}
    
    \section{Introduction}
    
    Photosynthesis represents the largest carbon flux between the atmosphere and land surface \shortcite{IPCC2013}, and plays a central role in biogeochemical cycling at multiple spatial and temporal scales \shortcite{Vitousek1991,LeBauer2008,Kaiser2015,Wieder2015}. Therefore, carbon and energy fluxes simulated by terrestrial biosphere models are sensitive to the formulation of photosynthetic processes \shortcite{Ziehn2011,Bonan2011,Booth2012,Smith2016,Smith2017} and must be represented using robust, empirically tested processes \shortcite{Prentice2015,Wieder2019}. Current formulations of photosynthesis vary across terrestrial biosphere models \shortcite{Smith2013,Rogers2017a}, which causes variation in modeled ecosystem processes \shortcite{Knorr2000,Knorr2001,Bonan2011,Friedlingstein2014} and casts uncertainty in ability of these models to accurately predict terrestrial ecosystem responses and feedbacks to global change \shortcite{Zaehle2005,Schaefer2012,Davies-Barnard2020}.

    Terrestrial biosphere models commonly represent C3 photosynthesis through variants of the \shortciteN{Farquhar1980} biochemical model \shortcite{Smith2013,Rogers2014,Rogers2017a}. This well-tested photosynthesis model estimates leaf-level carbon assimilation, or photosynthetic capacity, as a function of the maximum rate of Ribulose-1,5-bisphosphate carboxylase-oxygenase (Rubisco) carboxylation (\textit{V}\textsubscript{cmax}) and the maximum rate of Ribulose-1,5-bisphosphate (RuBP) regeneration (\textit{J}\textsubscript{max}; Farquhar et al., 1980). Many terrestrial biosphere models predict these model inputs based on plant functional group specific linear relationships between leaf nutrient content and (\textit{V}\textsubscript{cmax}) \shortcite{Smith2013,Rogers2014,Rogers2017a} under the tenet that a large fraction of leaf nutrients, and nitrogen (N) in particular, are partitioned toward building and maintaining enzymes that support photosynthetic capacity, such as Rubisco \shortcite{Brix1971,Gulmon1981,Evans1989,Kattge2009,Walker2014}. Terrestrial biosphere models also predict leaf nutrient content from soil nutrient availability based on the assumption that increasing soil nutrients generally increases leaf nutrients \shortcite{Firn2019}


    \section{Methods}
    
    \section{Results}