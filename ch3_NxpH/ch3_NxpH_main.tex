\begin{singlespace}
    \chapter{\textbf{Soil nitrogen availability modifies leaf nitrogen economies in mature temperate deciduous forests: a direct test of photosynthetic least-cost theory}}
    \end{singlespace}
    
    \section{Introduction}

    Photosynthesis represents the largest carbon flux between the atmosphere and land surface \shortcite{IPCC2021}, and plays a central role in biogeochemical cycling at multiple spatial and temporal scales \shortcite{Vitousek1991,LeBauer2008,Kaiser2015,Wieder2015_NPP}. Therefore, carbon and energy fluxes simulated by terrestrial biosphere models are sensitive to the formulation of photosynthetic processes \shortcite{Ziehn2011,Bonan2011,Booth2012,Smith2016,Smith2017} and must be represented using robust, empirically tested processes \shortcite{Prentice2015,Wieder2019}. Current formulations of photosynthesis vary across terrestrial biosphere models \shortcite{Smith2013,Rogers2017a}, which causes variation in modeled ecosystem processes \shortcite{Knorr2000,Knorr2001,Bonan2011,Friedlingstein2014} and casts uncertainty on the ability of these models to accurately predict terrestrial ecosystem responses and feedbacks to global change \shortcite{Zaehle2005,Schaefer2012,Davies-Barnard2020}  

    Terrestrial biosphere models commonly represent C3 photosynthesis through variants of the \shortciteN{Farquhar1980} biochemical model \shortcite{Smith2013,Rogers2014,Rogers2017a}. This well-tested photosynthesis model estimates leaf-level carbon assimilation, or photosynthetic capacity, as a function of the maximum rate of Ribulose-1,5-bisphosphate carboxylase-oxygenase (Rubisco) carboxylation ($V_{\mathrm{cmax}}$) and the maximum rate of Ribulose-1,5-bisphosphate (RuBP) regeneration (Jmax; Farquhar et al., 1980). Many terrestrial biosphere models predict these model inputs based on plant functional group specific linear relationships between leaf nutrient content and Vcmax \shortcite{Smith2013,Rogers2014,Rogers2017a} under the tenet that a large fraction of leaf nutrients, and nitrogen (N) in particular, are partitioned toward building and maintaining enzymes that support photosynthetic capacity, such as Rubisco \shortcite{Brix1971,Gulmon1981,Evans1989,Kattge2009,Walker2014}. Terrestrial biosphere models also predict leaf nutrient content from soil nutrient availability based on the assumption that increasing soil nutrients generally increases leaf nutrients \shortcite{Firn2019,Li2020,Liang2020}, which, in the case of nitrogen, generally corresponds with an increase in photosynthetic processes \shortcite{Li2020,Liang2020}.

    Recent work calls the generality of relationships between soil nutrient availability, leaf nutrient content, and photosynthetic capacity into question, suggesting instead that leaf nutrients and photosynthetic capacity are better predicted as an integrated product of aboveground climate, leaf traits, and soil nutrient availability, rather than soil nutrient availability alone \shortcite{Dong2017,Dong2020,Dong2022a,Firn2019,Smith2019,Peng2021}. It has been reasoned that this result is because plants allocate added nutrients to growth and storage rather than alterations in leaf chemistry \shortcite{Smith2019}, perhaps as a result of nutrient limitation of primary productivity \shortcite{LeBauer2008,Fay2015}. Additionally, recent work suggests that relationships between leaf nutrient content and photosynthesis vary across environments, and that the proportion of leaf nutrient content allocated to photosynthetic tissue varies over space and time with plant acclimation and adaptation responses to light availability, vapor pressure deficit, soil pH, soil nutrient availability, and environmental factors that influence leaf mass per area \shortcite{Pons1994,Niinemets1997,Evans2001,Hikosaka2009,Ghimire2017,Onoda2017,Luo2021}. The use of linear relationships between leaf nutrient content and $V_{\mathrm{cmax}}$ to predict photosynthetic capacity, as commonly used in terrestrial biosphere models \shortcite{Rogers2014}, is not capable of detecting such responses.


    \section{Methods}
    
    \section{Results}