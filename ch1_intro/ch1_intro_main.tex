\begin{singlespace}
    \chapter{\textbf{Introduction}}
\end{singlespace}

Photosynthesis represents the largest carbon flux between the atmosphere and biosphere, and is regulated by complex ecosystem carbon and nutrient cycles \shortcite{Hungate2003,IPCC2021}. As a result, the inclusion of robust, empirically tested representations of photosynthetic processes is critical in order for terrestrial biosphere models to accurately and reliably simulate carbon and nutrient fluxes between the atmosphere and terrestrial biosphere \shortcite{Oreskes1994,Smith2013,Prentice2015,Wieder2015_NPP}. Despite evidence that the inclusion of coupled carbon and nutrient cycles can improve model uncertainty, widespread divergence in predicted carbon and nutrient fluxes is still apparent across model products \shortcite{Friedlingstein2014,Arora2020,Davies-Barnard2020}. Divergence in predicted carbon and nutrient fluxes across terrestrial biosphere models may be due to an incomplete understanding of how plants acclimate to changing environments \shortcite{Smith2013,Davies-Barnard2020}, as terrestrial biosphere models are sensitive to the formulation of photosynthetic processes \shortcite{Bonan2011,Ziehn2011,Booth2012,Smith2016,Smith2017,Rogers2017a}.

Many terrestrial biosphere models predict leaf-level photosynthesis through linear relationships between area-based leaf nitrogen content and the maximum rate of Ribulose-1,5-bisphosphate carboxylase/oxygenase (``Rubisco''), following the idea that large fractions of leaf nitrogen content are allocated to the construction and maintenance of Rubisco and other photosynthetic enzymes \shortcite{Evans1989_photoN}. The inclusion of coupled carbon and nutrient cycles in terrestrial biosphere models \shortcite{Shi2016,Braghiere2022} allows for the prediction of leaf nitrogen content through soil nitrogen availability, which causes models to indirectly predict photosynthetic processes through shifts in soil nitrogen availability \shortcite{Smith2014,Lawrence2019}. While these patterns are commonly observed in ecosystems globally \shortcite{Brix1971,Evans1989_photoN,Firn2019,Liang2020}, this formulation does not allow for the prediction of leaf and whole plant acclimation responses to changing environments \shortcite{Smith2013,Rogers2017a,Harrison2021}, and suggests that constant leaf nitrogen-photosynthesis relationships are ubiquitous across ecosystems. 

Photosynthetic least-cost theory \shortcite{Prentice2014,Wang2017,Smith2019,Paillassa2020,Scott2022,Harrison2021} provides a contemporary framework for predicting leaf and whole plant acclimation responses to environmental change. The theory, which unifies photosynthetic optimal coordination \shortcite{Chen1993,Maire2012} and least-cost \shortcite{Wright2003} theories, posits that plants optimize photosynthetic processes by minimizing the summed cost of nutrient and water use (i.e., $\beta$). The summed cost of nutrient and water use is predicted to be positively correlated with the ratio of intercellular CO$_2$ to atmospheric CO$_2$ (leaf $C_\mathrm{i}$:$C_\mathrm{a}$). Leaf $C_\mathrm{i}$:$C_\mathrm{a}$ is determined by factors that influence leaf nutrient demand, such as CO$_2$, temperature, vapor pressure deficit, and light availability \shortcite{Prentice2014,Wang2017,Smith2019,Stocker2020}, and may change in response to changing edaphic characteristics through changes in $\beta$ \shortcite{Paillassa2020}. Photosynthetic processes are optimized such that nutrients and water are allocated to photosynthetic enzymes to allow net photosynthesis rates to be equally co-limited by the maximum rate of Rubisco carboxylation and the maximum rate of Ribulose-1,5-bisphosphate (RuBP) regeneration \shortcite{Chen1993,Maire2012}. The theory indicates that costs of nutrient and water use are substitutable such that, in a given environment, optimal photosynthesis rates can be achieved by sacrificing inefficient use of a relatively more abundant (and less costly to acquire) resource for more efficient use of a relatively less abundant (and more costly to acquire) resource.

Optimality models leveraging patterns expected from photosynthetic least-cost theory have been developed for both C$_3$ \shortcite{Wang2017,Smith2019,Stocker2020} and more recently for C$_4$ species \shortcite{Scott2022}. Such models show broad agreement with patterns observed across environmental gradients \shortcite{Smith2019,Stocker2020,Paillassa2020,Querejeta2022,Westerband2023}, and are capable of reconciling dynamic leaf nitrogen-photosynthesis relationships and acclimation responses to elevated CO$_2$, temperature, light availability, and vapor pressure deficit \shortcite{Dong2017,Dong2020,Smith2020,Luo2021,Peng2021,Dong2022a,Dong2022_eCO2,Querejeta2022,Westerband2023}. Current versions of optimality models that invoke patterns expected from photosynthetic least-cost theory hold $\beta$ constant across growing environments. As growing evidence suggests that costs of nutrient use change across resource availability and climatic gradients in species with different nutrient acquisition strategies \shortcite{Fisher2010,Brzostek2014FUN2,Terrer2018,Allen2020}, one might expect that $\beta$ should dynamically change across environments and in species with different nutrient acquisition strategies.

Despite recent recognition that patterns expected from photosynthetic least-cost theory occur across broad environmental gradients, a limited number of studies have investigated how $\beta$ varies across edaphic and climatic gradients and how variance in $\beta$ might scale to influence leaf nutrient-water use tradeoffs \shortcite{Lavergne2020,Paillassa2020}. Furthermore, no previous study has investigated whether $\beta$ varies in species with different nutrient acquisition strategies, or if changes in $\beta$ due to changes in edaphic characteristics scale to influence leaf or whole plant acclimation responses to changing environments. The lack of such studies provided motivation for the experimental chapters included in this dissertation.

In this dissertation, I use a combination of greenhouse, field manipulation, environmental gradient, and growth chamber experiments to quantify leaf and whole plant acclimation responses across various climatic and edaphic conditions and different nutrient acquisition strategies. Together, these experiments evaluate patterns expected from photosynthetic least-cost theory and test mechanisms predicted to drive responses expected from theory. The empirical data collected in these experiments provide important information needed to refine existing optimality models that include photosynthetic least-cost frameworks, and could help determine whether such models are suitable for implementing in next-generation terrestrial biosphere models. While theory suggests that plants acclimate across environments by minimizing the summed cost of nutrients relative to water, I choose to isolate effects of soil nitrogen availability on costs of nitrogen acquisition relative to water for the sake of brevity. I acknowledge that patterns expected from theory may be modified by other nutrients (e.g., phosphorus) or other edaphic characteristics, and, though not included here, should be investigated.

In the first experimental chapter, I re-analyze data from a greenhouse experiment that grew \textit{Glycine max} and \textit{Gossypium hirsutum} seedlings under full-factorial combinations of four light treatments and four fertilization treatments to examine effects of nitrogen and light availability on structural carbon costs to acquire nitrogen. In the second experimental chapter, I measure leaf physiological traits in the upper canopy of mature trees growing in a 9-year nitrogen-by-pH field manipulation experiment to assess whether changes in soil nitrogen availability or soil pH modify nitrogen-water use trade-offs expected from photosynthetic least-cost theory. The third experimental chapter leverages a broad precipitation and soil nutrient availability gradient in Texan grasslands to investigate primary drivers of leaf nitrogen content. In the fourth experimental chapter, I use growth chambers to quantify leaf and whole plant acclimation responses to CO$_2$ across a soil nitrogen fertilization gradient, while also manipulating nutrient acquisition strategy by controlling whether seedlings were able to form associations with symbiotic nitrogen-fixing bacteria.

Across experiments, I find strong and consistent support for patterns expected from photosynthetic least-cost theory, showing that shifts in edaphic characteristics predictably alter $\beta$, and that shifts in $\beta$ facilitate changes in leaf nitrogen-water use tradeoffs and leaf nitrogen-photosynthesis relationships. I also show that costs of nitrogen acquisition vary in species with different nitrogen acquisition strategies. Finally, I show strong evidence suggesting that leaf acclimation responses to elevated CO$_2$ are decoupled from soil nitrogen availability and inoculation with symbiotic nitrogen-fixing bacteria. It is my hope that these experiments will encourage future iterations of optimality models that adopt photosynthetic least-cost frameworks to consider frameworks for implementing dynamic $\beta$ values across soil resource availability gradients and in species with different nutrient acquisition strategies.

The four experimental chapters presented in this dissertation are presented either as previously published journal articles or as manuscript drafts currently in preparation for journal submission. Specifically, the first experimental chapter was published in \textit{Journal of Experimental Botany} in 2021 and the second chapter is currently in review, while the third and fourth chapters are each in preparation for journal submission. This dissertation concludes with a sixth chapter that summarizes experiment findings and briefly synthesizes common themes observed across experiments.