\begin{singlespace}
    \chapter{\textbf{Introduction}}
\end{singlespace}

Photosynthesis represents the largest carbon flux between the atmosphere and biosphere, and is regulated by complex ecosystem carbon and nutrient cycles \shortcite{Hungate2003,IPCC2021}. As a result, the inclusion of robust, empirically tested representations of photosynthetic processes is critical in order for terrestrial biosphere models to accurately and reliably simulate carbon and nutrient fluxes between the atmosphere and terrestrial biosphere \shortcite{Wieder2015_NPP,Smith2013,Prentice2015,Oreskes1994}. Despite evidence that the inclusion of coupled carbon and nutrient cycles can improve model uncertainty \shortcite{Arora2020,Braghiere2022,Shi2016}, widespread divergence in predicted carbon and nutrient fluxes is still apparent across model products \shortcite{Friedlingstein2014,Davies-Barnard2020}. This divergence could be due to an incomplete understanding of how plants acclimate to changing environments, particularly in response to shifts in soil resource avaialbility or aboveground climate \shortcite{Smith2013,Davies-Barnard2020}.

Photosynthetic least-cost theory \shortcite{Prentice2014,Wang2017,Smith2019,Scott2022} provides a contemporary framework for incorporating plant acclimation responses to changing environments in terrestrial biosphere models by unifying photosynthetic optimal coordination \shortcite{Chen1993,Maire2012} and least-cost \shortcite{Wright2003} theories. First principles of the theory predict that plants acclimate to a given environment by minimizing the summed costs of nutrient and water use in such a way that maximizes the use of available light and allows photosynthesis to be optimized by equal colimitation of Ribulose-1,5-bisphosphate carboxylase-oxygenase ("Rubisco") carboxlation and Ribulose-1,5-bisphosphate ("RuBP") regeneration rates. Optimality models leveraging patterns expected from photosynthetic least-cost theory have been developed \shortcite{Wang2017,Smith2019,Stocker2020,Scott2022}, though empirical tests of the theory are sparse.

In the time since designing experiments for this dissertation, environmental gradient studies have shown broad support for patterns expected from theory \shortcite{Paillassa2020,Querejeta2022,Westerband2023}, and the theory has been shown to be capable of predicting leaf across environmental changes such as atmospheric CO$_2$, temperature, growing season irradiance, and vapor pressure deficit \shortcite{Peng2021,Dong2017,Dong2020,Dong2022a,Dong2022_eCO2,Luo2021,Smith2020}. Specifically, the theory is capable of simulating the reduction in leaf nitrogen content and photosynthetic capacity with increasing CO$_2$, strong nitrogen-water use tradeoffs in response to changing temperature and vapor pressure deficit, and strong nitrogen-water use tradeoffs in response to shifts in soil nutrient availability and soil moisture. However, targetted experiments that test underlying mechanisms of the theory, particularly in response to changing soil resource availability and interactions between soil resource availability and aboveground climatic factors, are needed to evaluate the generality of patterns expected from theory and determine whether the theory is suitable for implementation in future generations of terrestrial biosphere models.

In this dissertation, I conduct a nitrogen-by-light manipulative greenhouse experiment, a nitrogen-by-sulfur manipulative field experiment, a soil resource avaialbility and climate environmental gradient field experiment, and a CO\textsubscript{2}-by-inoculation-by-nitrogen manipulative growth chamber experiment to test underlying assumptions of photosynthetic least-cost theory.