\begin{singlespace}
    \chapter{\textbf{Introduction}}
\end{singlespace}


Terrestrial ecosystems are regulated by complex carbon and nutrient cycles. As a result, terrestrial biosphere models, which are beginning to include linked carbon and nutrient cycles \shortcite{Shi2016,Davies-Barnard2020,Braghiere2022}, must accurately represent these cycles under different environmental scenarios to reliably simulate carbon and nitrogen fluxes between the atmosphere and terrestrial biosphere fluxes \shortcite{Oreskes1994,Hungate2003,Prentice2015}. While the inclusion of coupled carbon and nitrogen cycles tends to reduce model uncertainty \shortcite{Arora2020}, carbon and nutrient flux simulations across terrestrial biosphere models tends to diverge under future environmental scenarios \shortcite{Friedlingstein2014,Meyerholt2020}. The widespread divergence of terrestrial biosphere model simulations may be driven by uncertainty in the response of photosynthetic processes across resource availability gradients and in response to environmental change. This is because photosynthesis is the largest carbon flux between the atmosphere and terrestrial biosphere, and is constrained by ecosystem carbon and nutrient cycles \shortcite{Hungate2003,IPCC2021,LeBauer2008,Fay2015}. Yet, open questions remain regarding the influence of soil resource availability and climate on plant nutrient acquisition, plant nutrient allocation, photosynthetic processes, and whole plant growth.



Here, I conduct a nitrogen-by-light manipulative greenhouse experiment, a nitrogen-by-sulfur manipulative field experiment, a soil resource avaialbility and climate environmental gradient field experiment, and a CO\textsubscript{2}-by-inoculation-by-nitrogen manipulative growth chamber experiment to test underlying assumptions of photosynthetic least-cost theory. Specifically, these experiments 


test effects of soil resource availability and aboveground climate on plant nutrient acquisition, plant nutrient allocation, photosynthetic processes, and whole plant growth

In this dissertation, I test underlying assumptions of photosynthetic least-cost theory. Using a greenhouse nitrogen-by-light manipulation experiment, I show that 


conducted a series of experiments to quantify effects of aboveground climate and soil resource availability on nutrient acquisition and alloc

Here, I propose a series of experiments to quantify nutrient acquisition and allocaton responses to resource availability gradients 
dissertation designed to quantify nutrient acquisition and allocation responses to varying environmental conditions and resource availability gradients through the lens of the least-cost theory. Specifically, I will address five main questions:



