\begin{singlespace}
    \chapter{\textbf{Introduction}}
\end{singlespace}

Photosynthesis represents the largest carbon flux between the atmosphere and biosphere, and is regulated by complex ecosystem carbon and nutrient cycles \shortcite{Hungate2003,IPCC2021}. As a result, the inclusion of robust, empirically tested representations of photosynthetic processes is critical in order for terrestrial biosphere models to accurately and reliably simulate carbon and nutrient fluxes between the atmosphere and terrestrial biosphere \shortcite{Wieder2015_NPP,Smith2013,Prentice2015,Oreskes1994}. Despite evidence that the inclusion of coupled carbon and nutrient cycles can improve model uncertainty, widespread divergence in predicted carbon and nutrient fluxes is still apparent across model products \shortcite{Arora2020,Friedlingstein2014,Davies-Barnard2020}. Divergence in predicted carbon and nutrient fluxes across terrestrial biosphere models may be due to an incomplete understanding of how plants acclimate to changing environments \shortcite{Smith2013,Davies-Barnard2020}, as terrestrial biosphere models are sensitive to the formulation of photosynthetic processes \shortcite{Ziehn2011,Bonan2011,Booth2012,Smith2016,Smith2017,Rogers2017a}.

Many terrestrial biosphere models predict leaf-level photosynthesis through linear relationships between area-based leaf nitrogen content and the maximum rate of Ribulose-1,5-bisphosphate carboxylase/oxygenase ("Rubisco"), following from the idea that large fractions of nitrogen allocated to leaf tissue are allocated to the construction and maintenance of Rubisco \shortcite{Evans1989_photoN}. The inclusion of coupled carbon and nutrient cycles in terrestrial biosphere models \shortcite{Shi2016,Braghiere2022} allows for the prediction of leaf nitrogen content through soil nitrogen availability, which causes models to indirectly predict photosynthetic processes through shifts in soil nitrogen availability \shortcite{Smith2014,Lawrence2019}. While these patterns are commonly observed in ecosystems globally \shortcite{Brix1971,Evans1989_photoN,Liang2020,Firn2019}, this formulation of photosynthetic processes does not allow for the prediction of leaf and whole plant acclimation responses to changing environments \shortcite{Smith2013,Rogers2017a,Harrison2021}. Incoporating leaf and whole plant acclimation schemes in terrestrial biosphere models is important \shortcite{Smith2013}, particularly because recent work indicates that variance in leaf nitrogen content and leaf photosynthesis across environmental gradients may be better explained as an integrated product of leaf acclimation responses to changing climates and soil nitrogen availability than soil nitrogen availability alone \shortcite{Dong2017,Dong2020,Smith2019,Querejeta2022,Dong2022a,Westerband2023}.

Photosynthetic least-cost theory \shortcite{Prentice2014,Wang2017,Smith2019,Scott2022,Harrison2021} provides a contemporary framework for predicting leaf and whole plant acclimation responses to environmental change. The theory, which unifies photosynthetic optimal coordination \shortcite{Chen1993,Maire2012} and least-cost \shortcite{Wright2003} theories, posits that plants optimize photosynthetic processes by minimizing the summed cost of nitrogen and water use (referred to here and in the rest of this dissertation as $\beta$). Photosynthetic processes are optimized such that nitrogen is allocated to photosynthetic enzymes in to allow net photosynthesis rates to be equally co-limited by the maximum rate of Rubisco carboxylation and the maximum rate of Ribulose-1,5-bisphosphate (RuBP) regeneration \shortcite{Chen1993,Maire2012}. The theory indicates that costs of nitrogen and water use are substitutable such that, in a given environment, optimal photosynthesis rates can be acheived by sacrificing inefficient use of a relatively more abundant (and less costly to acquire) resource for more efficient use of a relatively less abundant (and more costly to acquire) resource. These predictions imply that acclimation responses to changing environments may be partially driven by tradeoffs between nitrogen and water use, though empirical tests of the theory are sparse.

Optimality models leveraging patterns expected from photosynthetic least-cost theory have been developed for both C$_3$ \shortcite{Wang2017,Smith2019,Stocker2020} and more recently for C$_4$ species \shortcite{Scott2022}. Such models show broad agreement with patterns observed across environmental gradients \shortcite{Paillassa2020,Querejeta2022,Smith2019,Westerband2023}, and are capable of reconciling dynamic leaf nitrogen-photosynthesis relationships and acclimation responses to elevated CO$_2$, temperature, light availability, and vapor pressure deficit \shortcite{Dong2017,Dong2020,Dong2022a,Dong2022_eCO2,Smith2020,Luo2021,Peng2021,Querejeta2022,Westerband2023}. Current versions of optimality models that invoke patterns expected from photosynthetic least-cost theory hold $\beta$ constant across growing environments. As growing evidence suggests that costs of nitrogen use change across resource availability and climatic gradients in species with different nutrient acquisition strategies \shortcite{Fisher2010,Brzostek2014FUN2,Allen2020,Terrer2018}, one might expect that $\beta$ should dynamically change across environments and in species with different acquisition strategies. However, manipulative experiments that test mechanisms underlying nitrogen-water use tradeoffs and leaf nitrogen-photosynthesis relationships predicted form theory are rare, and no study has related these patterns to shifts in $\beta$ or across species with different nutrient acquisition strategies. Understanding the dynamicism of $\beta$ across different environmental contexts and impacts of $\beta$ on patterns expected from theory are critical for further optimality model development, and is the central motivation for the experiments presented in this dissertation.

In this dissertation, I use four experiments to quantify nutrient acquisition and allocation responses under different environmental conditions and in species with different nutrient acquisition strategies. These experiments provide important empirical data needed to evaluate patterns expected from photosynthetic least-cost theory and test mechanisms that drive such patterns. In the first experimental chapter, I re-analyze data from a greenhouse experiment that grew \textit{Glycine max} L. (Merr) and \textit{Gossypium hirsutum} seedlings under full-factorial cominations of four light treatments and four fertilization treatments. This re-analysis examined the effect of soil nitrogen avaialbility and light availability on structural carbon costs to acquire nitrogen in a species capable of forming associations with symbiotic nitrogen-fixing bacteria (\textit{G. max}) and a species not capable of forming such associations (\textit{G. hirsutum}). I find strong evidence suggesting that increasing light availability increases structural carbon costs to acquire nitrogen and that increasing soil nitrogen fertilization decreases structural carbon costs to acquire nitrogen.

In the second experimental chapter, I measure leaf physiological traits in the upper canopy of mature trees growing in a 9-year nitrogen-by-pH field manipulation experiment to assess whether changes in soil nitrogen availability or soil pH modify nitorgen-water use tradeoffs expected from photosynthetic least-cost theory. I find strong nitrogen-water use tradeoffs in response to increasing soil nitrogen availability, indicated by a strong negative relationship between leaf $C_\mathrm{i}$:$C_\mathrm{a}$ (referred to here and in the rest of this dissertation as $\chi$) and leaf nitrogen content, as well as a strong increase in leaf nitrogen content per unit leaf $\chi$ with increasing soil nitrogen availability. Interestingly, I also find a null effect of soil pH on nitrogen-water use tradeoffs. These patterns provide strong support for patterns expected from photosynthetic least-cost theory across soil nitrogen availability gradients, and indicate that previous studies which note strong nitrogen-water use tradeoffs in response to soil pH may be driven by covariation between soil nitrogen availability and soil pH \shortcite{Paillassa2020,Westerband2023}.

In the third experimental chapter, I leverage a broad precipitation and soil nutrient availability gradient in Texan grasslands to investigate primary drivers of leaf nitrogen content. In this chapter, I directly quantify $\beta$ and $\chi$ using leaf $\delta^{13}$C to examine primary drivers of leaf nitrogen content and find that leaf nitrogen content is driven through a negative relationship with $\chi$. I also show that soil nitrogen availability is negatively associated with $\beta$, and that $\beta$ is positively associated with $\chi$. I show strong support for patterns expected from theory, showing for the first time that positive effects of increasing soil nitrogen availability on leaf nitrogen content are mediated by changes in $\beta$.

In the fourth experimental chapter, I use reach-in growth chambers to quantify leaf and whole plant acclimation responses to CO$_2$ across a soil nitrogen fertilization gradient, while also manipulating nutrient acquisition strategy by controlling whether seedlings were able to form associations with symbiotic nitrogen-fixing bacteria. Specifically, I measure leaf physiological and whole plant growth responses of 7-week \textit{G. max} seedlings grown under one of two CO$_2$ treatments, one of nine fertilization treatments, and one of two inoculation treatments in a full factorial design. I find a downregulation in leaf nitrogen content and leaf photosynthesis under elevated CO$_2$, a pattern that is not modified across the fertilization gradient or between inoculation treatments. However, I also find strong stimulations in total leaf area and whole plant growth under elevated CO$_2$ that are enhanced with increasing fertilization. There was no observable effect of inoculation in modifying whole plant growth responses to CO$_2$, which I speculate is the result of a downregulation in plant investments to nitrogen fixation with increasing fertilization. Results from this experiment provid strong evidence suggesting that leaf acclimation responses to CO$_2$ were controlled by optimal resource investment to photosynthetic capacity, following patterns expected from theory, and suggest divergent roles of soil nitrogen fertilization in modifying leaf and whole plant acclimation responses to CO$_2$.

Throughout the four experimental chapters, I find strong and consistent patterns that are supportive of patterns expected from photosynthetic least-cost theory. Specifically, I find strong nitrogen-water use tradeoffs in response to changing climates and soil resources, and that shifts in soil nitrogen availability have strong negative impacts on costs of nitrogen acquisition, and therefore tend to increase $\beta$. In a final conclusion chapter, I summarize major findings from each of the four experimental chapters and synthesize common mechanisms that drive leaf and whole plant responses to changing environmental conditions. I conclude this dissertation with brief dialogue on lessons learned throughout the experimental chapters, and propose future experiments that will target additional uncertainties in photosynthetic least-cost theory responses across environmental gradients.