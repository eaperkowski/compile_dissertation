\begin{singlespace}
    \chapter{\textbf{The relative cost of resource use for photosynthesis drives variance in leaf nitrogen content across climate and soil resource availability gradients}}
    \end{singlespace}
    
    \section{Introduction}
    
    Terrestrial biosphere models, which comprise the land surface component of Earth system models, are sensitive to the formulation of photosynthetic processes \shortcite{Knorr2000,Ziehn2011,Booth2012}. This is because photosynthesis is the largest carbon flux between the atmosphere and terrestrial biosphere, and is constrained by ecosystem carbon and nutrient cycles \shortcite{Hungate2003,LeBauer2008,IPCC2021,Fay2015}. Many terrestrial biosphere models formulate photosynthesis by parameterizing photosynthetic capacity within plant functional groups through empirical linear relationships between area-based leaf nitrogen content ($N_\mathrm{area}$) and the maximum carboxylation rate of Ribulose-1,5-bisphosphate carboxylase/oxygenase \shortcite{Kattge2009,Rogers2014,Rogers2017a}. Models are also beginning to include connected carbon-nitrogen cycles \shortcite{Wieder2015_NPP,Shi2016,Davies-Barnard2020,Braghiere2022}, which allows leaf photosynthesis to be predicted directly through changes in $N_{\mathrm{area}}$ and indirectly through changes in soil nitrogen availability (e.g., LPJ-GUESS, Smith et al., 2014; CLM5.0, Lawrence et al., 2019). Despite recent model developments, open questions remain regarding the generality of ecological relationships between soil nitrogen availability, leaf nitrogen content, and leaf photosynthesis across edaphic and climatic gradients.

    Empirical support for positive relationships between soil nitrogen availability and $N_\mathrm{area}$ is abundant \shortcite{Firn2019,Liang2020}, and is a result often attributed to the high nitrogen cost of building and maintaining Rubisco \shortcite{Evans1989_photoN,EvansSeemann1989,Onoda2004,Onoda2017,Dong2020}. Such patterns imply that positive relationships between soil nitrogen availability and $N_\mathrm{area}$ should cause an increase in leaf photosynthesis and photosynthetic capacity by increasing the maximum rate of Rubisco carboxylation through increased investments to Rubisco construction and maintenance. This integrated $N_\mathrm{area}$-photosynthesis response to soil nitrogen availability has been observed both in manipulative experiments and across environmental gradients \shortcite{Field1986,Evans1989_photoN,Walker2014,Li2020}, and is thought to be driven by ecosystem nitrogen limitation, which limits primary productivity globally \shortcite{LeBauer2008,Fay2015}. However, this response is not consistently observed, as recent studies note variable $N_\mathrm{area}$-photosynthesis relationships across soil nitrogen availability gradients \shortcite{Liang2020,Luo2021} and that aboveground growing conditions (e.g., light availability, temperature, vapor pressure deficit) or species identity traits (e.g., photosynthetic pathway, nitrogen acquisition strategy) may be more important for explaining variance in $N_\mathrm{area}$ and photosynthetic capacity across time and space \shortcite{Adams2016,Dong2017,Dong2020,Dong2022a,Smith2019,Peng2021,Westerband2023}.

    \section{Methods}

    \subsection{textit{Site descriptions and sampling methodology}}

    We collected leaf and soil samples from 24 open grassland sites across central and eastern Texas in summer 2020 and summer 2021 (Fig. \ref{fig:figure4.1}). Twelve sites were visited between June and July 2020 and 14 sites (11 unique from 2020) were visited between May and June 2021 (Table 1). We explicitly chose sites that maximized variability in precipitation and edaphic variability between sites while minimizing temperature variability across the environmental gradient (Table 1). No site with personally communicated or anecdotal evidence of grazing or disturbance (e.g., mowing, feral hog activity, etc.) were used. We collected leaf material from three individuals each of the five most abundant species at random locations at each site, only  selecting species that were broadly classified as graminoid, forb/herb, shrub, or subshrub growth habits per the USDA PLANTS database \shortcite{USDANRCS2022}. All collected leaves were fully expanded with no visible herbivory or other external damage and also free from shading by nearby shrubs or trees. Five soil samples were collected from 0-15cm below the soil surface at each site near the leaf collection sample locations. Soil samples were later mixed together by hand to create one composite soil sample per site.

    \subsection{Leaf trait measurements}
    Images of each leaf were taken immediately following each site visit using a flat-bed scanner. Fresh leaf area was determined from each image using the 'LeafArea' R package \shortcite{Katabuchi2015}, which automates leaf area calculations using ImageJ software \shortcite{Schneider2012}. Each leaf was dried at 65\textdegree{}C for at least 48 hours to a constant mass, weighed, and manually ground in a mortar and pestle until homogenized. Leaf mass per area ($M_\mathrm{area}$; $\mathrm{g\ m^{-2}}$) was calculated as the ratio of dry leaf biomass to fresh leaf area. Subsamples of dried and homogenized leaf tissue were used to measure leaf nitrogen content ($N_\mathrm{mass}$; $\mathrm{gN\ g^{-1}}$) through elemental combustion analysis (Costech-4010, Costech Instruments, Valencia, CA). Leaf nitrogen content per unit leaf area ($N_area$; $\mathrm{gN\ m^{-2}}$) was then calculated as the product of $N_\mathrm{mass}$ and $M_\mathrm{area}$.
    
    Subsamples of dried and homogenized leaf tissue were sent to the University of California-Davis Stable Isotope Facility to determine leaf $\delta^{13}$C. Leaf $\delta^{13}$C values were determined using an elemental analyzer (PDZ Europa ANCA-GSL; Sercon Ltd., Chestshire, UK) interfaced to an isotope ratio mass spectrometer (PDZ Europa 20-20 Isotope Ratio Mass Spectrometer, Sercon Ltd., Chestshire, UK). We used leaf $\delta^{13}$C values (‰; relative to Vienna Pee Dee Belemnite international reference standard) to estimate the ratio of intercellular ($C_\mathrm{i}$) to extracellular ($C_\mathrm{a}$) CO\textsubscript{2} ratio (leaf $C_\mathrm{i}$:$C_\mathrm{a}$, $\chi$; unitless) following the approach of Farquhar et al. (1989) described in Cernusak et al. (2013). We derived $\chi$ as:

    \begin{equation} 
        \label{eq_4.1}
        \chi=\frac{C_{i}}{C_{a}}=\frac{\Delta^{13}C - a}{b - a}
    \end{equation}
    
    \noindent where $\Delta^{13}$C represents the relative difference between leaf $\delta^{13}$C (‰) and air $\delta^{13}$C (‰), and is calculated as:

    \begin{equation}
        \label{eq_4.2}
        \Delta^{13}C = \frac{\delta^{13}C_{air} - \delta^{13}C_{leaf}}{1 + \delta^{13}C_{leaf}}
    \end{equation}

    \noindent $\delta^{13}\mathrm{C_{air}}$, traditionally assumed to be -8‰ \shortcite{Keeling1979,Farquhar1989}, was calculated as a function of calendar year \textit{t} using an empirical equation derived in \shortciteN{Feng1999}:

    \begin{equation}
        \label{eq_4.3}
        \delta^{13}C_{air} = -6.429 - 0.006e^{0.0217(t-1740)}
    \end{equation}
    
    \noindent This calculation resulted in $\delta^{13}\mathrm{C_{air}}$ values for 2020 and 2021 as -9.04 and -9.09, respectively. a represents the fractionation between $^{12}\mathrm{C}$ and $^{13}\mathrm{C}$ due to diffusion in air, assumed to be 4.4‰, and b represents the fractionation caused by Rubisco carboxylation, assumed to be 27‰ \shortcite{Farquhar1989}. For $C_{4}$ species, b in Eqn. \ref{eq_4.1} was set to 6.3‰, and was derived from:

    \begin{equation}
        \label{eq_4.4}
        b = c + (d \cdot \phi)
    \end{equation}
    
    \noindent Where c was set to -5.7‰ and d was set to 30‰ \shortcite{Farquhar1989}. $\phi$, which is the bundle sheath leakiness term, was set to 0.4. All $\chi$ values less than 0.2 and greater than 1.0 were assumed to be incorrect and removed.
    
    We derived the unit cost of resource use ($\beta$) using leaf $\chi$ and site climate data with equations first described in \shortciteN{Prentice2014} and simplified in \shortciteN{Lavergne2020}:

    \begin{equation}
        \label{eq_4.5}
        \beta = 1.6\eta^{*} D \frac{\chi - (\frac{\Gamma^*}{C_{a}})^{2}}{(1 - \chi)^{2}(K_{m} + \Gamma^{*})}
    \end{equation}
    
    where $\eta^{*}$ is the viscosity of water relative to 25\textdegree{}C, calculated using elevation and mean air temperature of the seven days leading up to each site visit following equations in \shortciteN{Huber2009}. D represents vapor pressure deficit (Pa), set to the mean vapor pressure deficit of the seven days leading up to each site visit, $C_\mathrm{a}$ represents atmospheric CO\textsubscript{2} concentration, arbitrarily set to 420 $\mathrm{\mu mol\ mol^{-1}}$ CO\textsuperscript{2}. $K_\mathrm{m}$ (Pa) is the Michaelis-Menten coefficient for Rubisco affinity to CO\textsubscript{2} and O\textsubscript{2}, calculated as:
    
    \begin{equation} \label{eq_4.6}
        K_{m} = K_{c} \cdot \left ( 1 + \frac{O_i}{K_o} \right )
    \end{equation}
    where $K_\mathrm{c}$ (Pa) and $K_\mathrm{o}$ (Pa) are the Michaelis-Menten coefficients for Rubisco affinity to CO\textsubscript{2} and O\textsubscript{2}, respectively, and $O_\mathrm{i}$  is the intercellular O\textsubscript{2} concentration. $\Gamma^{*}$ (Pa) is the CO\textsubscript{2} compensation point in the absence of dark respiration. $K_\mathrm{c}$, $K_\mathrm{o}$, and $\Gamma^{*}$ were determined using equations described in \shortciteN{Medlyn2002} and derived in \shortciteN{Bernacchi2001}, invoking an elevation correction for atmospheric pressure as explained in \shortciteN{Stocker2020}.

    \clearpage

    \newpage
    placeholder for Table 1
    \clearpage

    \newpage
    \begin{landscape}
        \begin{figure}
            \centering
            \includegraphics[scale = 0.05]{ch4_TXeco/figs/TXeco_fig1_site_map.png}
            \caption[Maps that detail site locations along 2006-2020 mean annual precipitation (panel A) and mean annual temperature (panel B) gradients in Texas, USA.]{Maps that detail site locations along 2006-2020 mean annual precipitation (panel A) and mean annual temperature (panel B) gradients in Texas, USA. Precipitation and temperature data were plotted at a 4-km grid resolution and are masked to include only grid cells that occur in the Texas state boundary in the United States. In both panels, open circles refer to sites visited in 2020, open triangles to sites visited in 2021, and closed circles to sites visited in 2020 and 2021. The scale bar in panel A also applies to panel B.}
            \label{fig:figure4.1}
        \end{figure}
    \end{landscape}
    \clearpage









    \section{Results}

    \section{Discussion}