\begin{singlespace}
    \chapter{\textbf{The relative cost of resource use for photosynthesis drives variance in leaf nitrogen content across climate and soil resource availability gradients}}
    \end{singlespace}
    
    \section{Introduction}

Terrestrial biosphere models, which comprise the land surface component of Earth system models, are sensitive to the formulation of photosynthetic processes \shortcite{Knorr2000,Ziehn2011,Booth2012}. This is because photosynthesis is the largest carbon flux between the atmosphere and terrestrial biosphere, and is constrained by ecosystem carbon and nutrient cycles \shortcite{Hungate2003,LeBauer2008,IPCC2021,Fay2015}. Many terrestrial biosphere models formulate photosynthesis by parameterizing photosynthetic capacity within plant functional groups through empirical linear relationships between area-based leaf nitrogen content ($N_{\mathrm{area}}$) and the maximum carboxylation rate of Ribulose-1,5-bisphosphate carboxylase/oxygenase \shortcite{Kattge2009,Rogers2014,Rogers2017a}. Models are also beginning to include connected carbon-nitrogen cycles \shortcite{Wieder2015_NPP,Shi2016,Davies-Barnard2020,Braghiere2022}, which allows leaf photosynthesis to be predicted directly through changes in$N_{\mathrm{area}}$ and indirectly through changes in soil nitrogen availability (e.g., LPJ-GUESS, Smith et al., 2014; CLM5.0, Lawrence et al., 2019). Despite recent model developments, open questions remain regarding the generality of ecological relationships between soil nitrogen availability, leaf nitrogen content, and leaf photosynthesis across edaphic and climatic gradients.